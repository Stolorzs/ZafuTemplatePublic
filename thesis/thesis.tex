% \documentclass[a4paper]{article}
% \usepackage{ctex}
\documentclass[AutoFakeBold]{ZafuThesis}

\academy{风景园林与建筑学院}%学院
\session{2024}%第*届
\author{买一盒、松手}%作者姓名
\studentNumber{20190000000000000}%学号
\myClass{建筑学191}%班级
\teacher{建筑系全体教师}{教授/副教授/讲师}%指导老师
\date{\today}%日期,显示在封面与诚信承诺书页
% \title{{浙江农林大学本科生毕业设计模板 }}%标题单行
\title{{浙江农林大学本科生}{毕业设计说明书或论文模板}}%标题双行
\entitle{Architectural design of Heshang Police Station of Xiaoshan District `Public Security Bureau}%英文标题,显示在摘要页

\begin{document}
% %封面
\customCover

%自己的签名
\setsignature
  {
  \includegraphics[width=100pt]{figures/MaiYihe}
  \includegraphics[width=120pt]{figures/SongShou}
  }
% 承诺页
\makestatement


%目录
\customContent

\frontmatter
%中文摘要,关键词分号分隔
\ZhAbstract
{
  摘要内容主要介绍所研究的课题内容、提出主要结论及创新之处。摘要部分格式:黑体加粗五号段前空两个汉字字符,摘要内容楷体五号,不超过300个汉字。关键词部分格式:黑体加粗五号段前空两个汉字字符,关键词内容楷体五号,术语用分号隔开,数量一般为3-6个。Abstract:Times New Roman加粗五号段前空两个汉字字符;Abstract内容:Times New Roman五号。Key Words: Times New Roman加粗五号段前空两个汉字字符; Key Words内容:Times New Roman五号,术语用逗号隔开。
  
}
{关键词1;关键词2;关键词3;关键词4;关键词5}


%英文摘要,关键词用逗号分隔
\EnAbstract
{In this short article we will discuss about LATEX for your dissertation}
{keyword 1, keyword 2, keyword 3}


\mainmatter
\section{使用说明}
ZafuTemplate(浙江农林大学本科生毕业设计说明/论文模板与开题报告模板)是厌倦了低效的Word的某人完成毕业任务时心血来潮所作,现发布到网上供浙江农林大学本科毕业生免费使用,在使用前请仔细阅读下面的内容和注意事项。\par
学校和学院的要求存在一定的差异,据了解,甚至同一学院的不同专业对论文模板的要求也有所不同(Zafu什么时候能统一一下啊!),且模板仍存在个别技术问题亟待解决。总之,本模版尚存不足之处,欢迎反馈,更希望大家能帮忙一起完善。
\subsection{免责声明}
本模板格式参照2024届毕设或论文材料要求内建筑学专业的示例文件所作。'.cls'格式的文件内容由本人原创,部分代码借鉴学习自兰州大学兰朵儿所作的本科毕业论文模板\href{https://github.com/yuhldr/LZUThesis2020}{
  ( \textcolor{blue}{点击可访问其GitHub页面})}。\par
作者的初衷是减少后人花费在格式调整上的无用功,使学弟学妹们能够专注于文章内容的写作,从而提高本科生的毕业论文/设计水平。作者不保证本模板完全符合学科要求,因使用本模板产生的损失{\bfseries 由使用者自负},作者{\bfseries 不承担任何责任}!
\subsection{目录下的文件说明}
\begin{itemize}
  \item ".vscode"是VSCode的配置文件(若使用别的编辑器可以忽略)
  \item "2024届毕设或论文材料要求"文件夹内包含学校要求说明与建筑学专业的示例文档
  \item "thesis"内是毕业设计说明/论文的LaTex模板{\bfseries 主体文件}
  \item "ZafuResearchProposal"内是开题报告的LaTex模板{\bfseries 主体文件}
\end{itemize}
\par
主题文件中'.tex' 文件是 LaTeX 文档的源文件。它包含实际的文档内容和 LaTeX 命令,用于生成最终的 PDF 文件;'.cls'文件是 LaTeX 文档类文件,定义了文档的整体布局和样式(如有需要,请在其中自行修改格式)。它包含了一组命令和宏,用于规范文档的格式;'figures'文件夹下为要在文中展现的图片;'.bib' 文件是用于存储参考文献的数据库文件。
\subsection{安装和配置}
请自行根据操作系统对映选择安装TeX发行版,在此不过多赘述。推荐使用VSCode编辑器编译运行,默认编译模式为xelatex。\par
教程推荐:\href{https://www.bilibili.com/video/BV12m4y1D7PZ/?vd_source=dfa6f0143619fda15de458493344dd04}{\textcolor{blue}{LaTeX论文写作指南——以VSCode编辑器为例}}\par
本模板主页:\href{https://github.com/Stolorzs/ZafuTemplatePublic}{\textcolor{blue}{ZafuTemplatePublic}}

\subsection{LaTex生成的PDF转化DOCX}
许多导师不会使用PDF编辑器从而要求学生提交DOCX批阅,或者学院要求提交DOCX格式的文档,迫于上述现实因素,不得不研究将LaTeX导出的PDF转化为DOCX格式的方法。\par
\paragraph{转化方式}
使用 Adobe Acrobat DC 打开 LaTeX 生成的 PDF 文件,然后选择“另存为 DOCX”即可完成转换。作者测试了 Adobe Acrobat DC 2023 及以上版本,绝大多数的字体格式与图片排版都能在Word中对映上,效果较好,但未对带公式的转化(因为咱建筑学写论文很少用公式)进行测试。
\paragraph{注意事项}
在 MacOS 下编译生成的PDF不要导出到Windows操作系统下转化,在 Windows 下编译生成的 PDF 也不要导出到 MacOS 操作系统下转化,不然字体的格式会发生错误。若转化效果欠佳,可以考虑将 PDF 拆分成多组内容,分组进行转化再合并。如,将毕业论文拆分为封面、诚信承诺书、目录、摘要、正文主体几部分内容,分组转化为DOCX后在Word中进行合并(Word不用说怎么用吧!使用Word的分页能实现\LaTeX 中\textbackslash clearpage的效果)。



\section{文字格式}
\subsection{参考规范}
请使用者自行参阅毕业当年的《浙江农林大学本科生毕业论文(设计)系列材料》与学科给出的示例文件。
\subsection{正文排版}
\LaTeX 是一种按照层级排列的标记语言,使用不同的符号来区分不同的内容,以正文部分为例,一个具有三层标题的正文内容片段主要由以下代码清单 \ref{list_ZWPD}所述组成。

\begin{lstlisting}[caption=正文片段, label = list_ZWPD,language = tex]
  \section{一级标题}  #花括号"{}"内撰写标题的内容
  正文第一段\par  #使用\par开启一个新的段落
  正文第二段
  \subsection{二级标题}
  正文第一段\par
  正文第二段
  \subsubsection{三级标题}
  正文第一段\par
  正文第二段
  #注意:1)标题的序号自动生成,无需填写序号。
  #      2)正文默认首行缩进两字符,如果想要某一段不具有首行缩进,可以在段首加上 \noindent
\end{lstlisting}
\par

\subsubsection{章节标题和序号}
章节标题和序号参考学校对于\textbf{理工科类}论文的格式要求,以阿拉伯数字为序号,采用数字和"."的组合。
\paragraph{一级标题}
采用楷体加粗四号居中,序号与文字间空1个汉字字符,段前、段后6磅(或1行);
\paragraph{二级标题}
采用黑体加粗小四,段前空两个汉字字符,序号与文字间空1个汉字字符,段前、段后3磅(或0.5行);
\paragraph{三级标题}
采用黑体五号,段前空两个汉字字符,序号与文字间空1个汉字字符,段前、段后3磅(或0.5行);
\paragraph{三级以下标题}
开题报告模板的四级标题使用 \textbackslash paragraph{四级标题} 命令。而毕业设计说明/论文模板的 \textbackslash paragraph{四级标题} 命令实现带有标题的段落格式,其四级标题尚未实现。(摆烂作者真的不想研究了!交给聪明的下一届好了!)\par 
模板使用titlesec宏包设置标题格式,以一级标题为例,具体设置如清单 \ref{list_一级标题设置}所示。
\begin{lstlisting}[caption=一级标题设置, label = list_一级标题设置,language = tex]
  \titleformat{\section}{\centering\zihao{4}\bfseries\CJKfamily{zhkai}}{\thesection}{0.5em}{}
  \titlespacing*{\section}{0pt}{9.75pt}{9.75pt}
\end{lstlisting}
\paragraph{注意}
关于titlesec宏包的具体用法请参阅:\href{https://static.latexstudio.net/wp-content/uploads/2016/12/titlesec_c.pdf}{\textcolor{blue}{titlesec文档(第5,6页)}}。Word与 \LaTeX 的度量单位(行间距等)并不完全一致,详见\href{https://www.bilibili.com/video/BV1jh4y197LW/?spm_id_from=333.337.search-card.all.click&vd_source=dfa6f0143619fda15de458493344dd04}{\textcolor{blue}{Latex排版巨坑——行间距}},如需转化,请调整时请合理转换。\par

\subsubsection{正文字体}
本模板使用宏包ctexart支持中文汉字。\par
\paragraph{默认样式}
根据学校规范,毕业设计说明/论文默认将正文字体设置为宋体五号,英文字体设置为Times New Roman,行间距20pt。(什么,你问默认的无衬线字体、等宽字体是什么样式?规范没说啊,就都按照正文字体来吧)\par
\paragraph{自定义字体}
ctexart宏包预设中文字体有四种:宋体、黑体、仿宋、楷书,具体命令如表 \ref{table_宏包预设字体}。
\begin{table}[htbp]
  \centering  
    \caption{宏包预设字体}
    \begin{tabular}{c c c c}
      \toprule[1pt]
      宋体 & 黑体 & 仿宋 & 楷书 \\
      \hline
      \textbackslash songti & \textbackslash heiti & \textbackslash fangsong & \textbackslash kaishu \\
      \toprule[1pt]
    \end{tabular}
    \label{table_宏包预设字体} 
\end{table}\par

\paragraph{自定义字号}
该宏包提供了控制字号的命令,例如\textbackslash zihao\{5\}为五号,\textbackslash zihao\{-4\}为小四,且仅当标准字体命令为\textbackslash normalsize 时有这样的对映结果,如表 \ref{table_字体命令对映字号}所示,具体关系请参阅:\href{https://mirrors.ibiblio.org/CTAN/language/chinese/ctex/ctex.pdf}{\textcolor{blue}{CTEX 宏集手册-5.1(p8)}}。使用ctexart宏包设置字体格式的具体方法请参阅:\href{https://mirrors.ibiblio.org/CTAN/language/chinese/ctex/ctex.pdf}{\textcolor{blue}{CTEX 宏集手册-第7节(p15)}}。

\begin{table}[htbp]
  \centering
    \caption{字体命令对映字号}
    \begin{tabular}{c c c c c}
      \toprule[1pt]
      & \multicolumn{2}{c|}{\textbackslash zihao\{5\}} & \multicolumn{2}{c}{\textbackslash zihao\{-4\}}  \\
      字体命令 & 字号 &\multicolumn{1}{c|}{bp}  & 字号 & bp \\
      \hline
      \textbackslash normalsize & 五号 & 10.5  &  小四 & 12 \\
      \toprule[1pt]
    \end{tabular}
    \label{table_字体命令对映字号}
  \end{table}

\begin{table}[htbp]
    \centering
    \caption{常用文字颜色}
    \begin{tabular}{c c c}
      \toprule[1pt]
      关键字 & 对映颜色\\
      \hline
      red &  \textcolor{red}{红色} \\
      blue &  \textcolor{blue}{蓝色} \\
      cyan &  \textcolor{cyan}{青色} \\
      green &  \textcolor{green}{绿色} \\
      purple &  \textcolor{purple}{紫色} \\
      \toprule[1pt]
    \end{tabular}
    \label{table_常用文字颜色}
\end{table}\par

\paragraph{自定义文字颜色}
使用 \textbackslash textcolor\{颜色\}\{文字\} 设置文字的颜色,其中颜色部分可以使用预设,也可以通过RGB色值设置自定义颜色: \textbackslash textcolor[rgb]\{0.25, 0.5, 0.75\}\{文字\}。表 \ref{table_常用文字颜色}展示了常用颜色预设与关键字的对映。

\paragraph{自定义字型}
\LaTeX 提供了设置字型的命令,常用字型设置(斜体、粗体)都对映两组全局与局部两组命令,如表 \ref{table_字型设置}所示。比如说,\textbackslash bfseries 为全局命令,或者写成 \{\textbackslash bfseries 内容 \} 这样的形式以实现局部定义字型;对映的局部命令为 \textbackslash textbf ,用于局部修改样式,需要写成 \textbackslash textbf\{内容\} 这样的形式。
\begin{table}[htbp]
  \centering
  \caption{字型设置}
  \begin{tabular}{c c c c}
    \toprule[1pt]
    全局命令 & 局部命令 & 英文 & 中文 \\
    \hline
    \textbackslash itshape  & \textbackslash textit\{...\} & \textit{italic}& \textit{意大利斜体} \\
    \textbackslash slshape  & \textbackslash textsl\{...\} & \textsl{slanted}& \textsl{倾斜体} \\
    \textbackslash bfseries  & \textbackslash textbf\{...\} & \textbf{Bold}& \textbf{粗体} \\
    \textbackslash mdseries  & \textbackslash textmd\{...\} & \textmd{medium}& \textmd{正常字体} \\
    \toprule[1pt]
    \label{table_字型设置}
  \end{tabular}
\end{table}

\paragraph{补充(开题报告)}
开题报告的模板参照自建筑学学科的示例文件编写。应学科要求,2024届正文要部分要带框。\par
为正文部分带框,需要保证内容在"RPSectionBox"环境下,其代码如\ref{list_开题报告带框} 所示。\par
如果使用者毕业一届又要求开题报告不带框,那就不要在"RPSectionBox"环境下填写内容。\textbf{注意},有框与无框情况下形成的字体行间距、段前段后、缩进的尺度并不完全相同,本模板仅保证带框环境下生成的格式与示例文件几乎一致。
\begin{lstlisting}[caption=开题报告带框, label = list_开题报告带框,language = tex]
\begin{RPSectionBox}
  \section{章节}
  内容...
  \subsection{章节}
  内容...
\end{RPSectionBox}
\end{lstlisting}\par
本模板提供的开题报告内容来自\textbf{松手},除上述带框环境的部分外,关于其具体的设置参考毕业设计说明/论文的说明即可,就(偷懒)不再写一份说明文档了。

\subsubsection{文献引用}
学科与学校对参考文献的引用格式并不统一,本模板按照\textbf{建筑学学科}对文内引用标注的要求,介绍两种常用方法。\par
文献引用的格式调用宏包gbt7714,实现中国的参考文献推荐标准GB/T 7714—2015《信息与文献参考文献著录规则》。说明详见该宏包的主页:\href{https://github.com/zepinglee/gbt7714-bibtex-style}{\textcolor{blue}{gbt7714-bibtex-style}}。\par
遗憾的是,Zafu建筑学学科的引用标准并不参照上述规范,但类似gbt7714提供的的“著者-出版年制参考文献表”的引用方式。文献引用页的具体条目格式有所区别,体现在标点符号与学科要求在条目前加上引用的索引。\par
虽然以下两种方法都能达到类似的效果,但是过程或者结果还不够完美,要实现完美的效果请移步学习bibTeX或natbib宏包(作者真的不想学啦!难题交给后人,加油)。
\begin{itemize}
  \item 方法1: 使用 \textbackslash bibitem\par
  \paragraph{参考文献页}
  根据学科提供的案例,参考文献页在正文文本后。在正文结束后,调用 \ref{list_bibitem}所示代码。
  \begin{lstlisting}[language = tex,caption = 使用\textbackslash bibitem,label = list_bibitem]
    \clearpage
    \bibliographystyle{gbt7714-numerical} %使用带索引的文献引用
    \begin{thebibliography}{99}
      \bibitem{ref1} reference1
      \bibitem{ref2} reference2
      \bibitem{ref3} reference3
    \end{thebibliography}
    %  reference1、reference2、reference3参考文献的引用条目,自行按照标准填写参考文献引用格式
  \end{lstlisting}

  \paragraph{文中引用}
  使用\textbackslash cite\{ref\}(ref就是 \textbackslash bibitem\{\} 中的参数)增加文内引用标注,其参考文献页条目的格式为带索引形式。
  \paragraph{优缺点}
  这种方法较为简单,简单改动就可以适应不同要求。但是文献引用不太智能,在文中引用部分只能手动输入如(作者, 年份)而且不支持跳转。

  \item 方法2: 使用.bib文件生成\par
  该方法需要选择编译方式:xe->bib->xe->xe。
  \paragraph{参考文献页}
  在.bib文件中加入文献的引用,然后在tex中自动生成,bib的具体内容可以在大多数文献的官网上导出。生成参考文献页的指令如清单 \ref{list_bibitem}所示。
  \begin{lstlisting}[language = tex,caption = 使用\textbackslash bibitem,label = list_bibitem]
    \clearpage
    %使用(作者, 年份)格式
    \bibliographystyle{gbt7714-author-year} 
    %表示引用ref.bib文件,用以生成参考文献页
    \bibliography{ref}
  \end{lstlisting}\par

  \paragraph{文中引用}
  使用\textbackslash citep\{code\}(code就是 .bib文件条目的编号)。
  \paragraph{优缺点}
  这种方式较为智能,不用在文中手动填写 作者, 年份 ,较为直观。但是,这样生成的参考文献是不带索引的。要带上索引,可以在PDF编辑器里编辑\LaTeX 生成的PDF。推荐使用福昕高级PDF编辑器(菜单-编辑-编辑文本)。注意调整字体的格式。
\end{itemize}
\par
本说明文件中生成的参考文献页使用了方法2所述的方式\citep{Teachers}\citep{MYH}\citep{SS}\citep{Students}。

\section{图片插入}
图片、表格等内容的插入,需要了解\LaTeX \textbf{浮动体}的基础知识,在此不做详细介绍。
\subsection{插入指令}
使用 \textbackslash includegraphics[]\{\} 命令,"[]"内的可选参数用于控制图片的尺寸,"\{\}"内填写图片的相对路径或绝对路径。\par
默认请将图片放在figures目录(自定义目录也不是不行)下。清单 \ref{list_简单图片插入}所示以简单图片的插入为例展示以上描述,图 \ref{figure_centeringFigure}是该代码生成的效果。
\begin{figure}[h]
  \centering\includegraphics[width=110pt]{figures/MaiYihe}
  \caption{这是一个居中的图片}
  \label{figure_centeringFigure}
\end{figure}
\begin{lstlisting} [language = tex,caption = 简单图片插入,label = list_简单图片插入]
  \section{图片插入}
  \begin{lstlisting}
    \begin{figure}[h]
      \centering      % \centering表示图表整体居中
      \includegraphics[width=100pt]{figures/MaiYihe}
      \caption{这是一个居中的图片} % 图片的脚注
    \end{figure}
  \end{lstlisting}\par

\subsection{图片排版}
截至2024年,Zafu的文件并未对图片的排版有明确规定(也就是可以随意发挥的意思咯?)。实际写作中,可能有图片竖排、并排、竖排结合并排等需求,在此仅作\textbf{简单介绍}。
\paragraph{图片竖排}
\LaTeX 图片排版的逻辑是:\textbf{同一页面}下的图片根据引用顺序从上到下排列。图 \ref{figures_sp0}和图 \ref{figures_sp1}展示了图片\textbf{竖排}的效果。\par
\begin{figure}[htbp]
  \centering
  \includegraphics[width=0.24\columnwidth]{figures/MaiYihe}
  \caption{竖排}
  \label{figures_sp0}
\end{figure}
\begin{figure}[htbp]
  \centering
  \includegraphics[width=0.26\columnwidth]{figures/SongShou}
  \caption{竖排}
  \label{figures_sp1}
\end{figure}

\paragraph{图片并排}
要实现并排的效果,可以有两种实现方式。一种是使用\textbf{子图} \textbackslash subfloat[]\{\}命令,若子图的尺寸得当,即子图宽度和不超过行宽,默认左右排序,如图 \ref{figure_子图实现并排}所示。\par
另一种方法是在图片的环境内,调用minipage环境创建\textbf{子页},在子页上放置\textbf{子图}。注意,子页上不能放置浮动体(图片也是浮动体),但是可以放置子图,如图 \ref{figure_子页中子图并排实现}所示。\par

\begin{figure}[htbp]
  \centering
  \subfloat[买一盒]
  {\includegraphics[width=0.28\columnwidth]{figures/MaiYihe}}
  \hspace{1cm}
  \subfloat[松手]
  {\includegraphics[width=0.30\columnwidth]{figures/SongShou}}
  \caption{子图实现并排}  
  \label{figure_子图实现并排}
\end{figure}


\begin{figure}[htbp]
  \centering
  \begin{minipage}[b]{0.45\columnwidth}
    \centering
    \subfloat
    {\includegraphics[width=0.6\textwidth]{figures/MaiYihe}}
  \end{minipage}
  \begin{minipage}[b]{0.45\columnwidth}
    \centering
    \subfloat
    {\includegraphics[width=0.6\textwidth]{figures/SongShou}}      
  \end{minipage}
  \caption{子页中子图并排实现}
  \label{figure_子页中子图并排实现}
\end{figure}

\paragraph{竖排结合并排}
竖排结合并排的排版方式是“子页中子图并排”的扩展。可以实现如图 \ref{figure_竖排结合并排}所示的效果。
\begin{figure}[htbp]
  \centering
  \begin{minipage}[b]{0.45\columnwidth}
    \centering
    \subfloat[]
    {\includegraphics[width=0.68\textwidth]{figures/zafu}}
    \\
    \subfloat[]
    {\includegraphics[width=0.68\textwidth]{figures/zafu}}
  \end{minipage}
  \begin{minipage}[b]{0.45\columnwidth}
    \centering
    \subfloat[]
    {\includegraphics[width=0.52\textwidth]{figures/zafuLogo}}      
  \end{minipage}
  \caption{竖排结合并排}
  \label{figure_竖排结合并排}
\end{figure}

\subsection{图片引用}
图片环境中与子图中的 \textbackslash label命令表示标签,可以通过 \textbackslash ref实现在文中的交叉引用。注意:首次添加的交叉引用,其内容在使用xelatex编译以后生成“??”,重新使用 xelatex 编译即可(即首次添加时xelatex->xelatex编译两次)。

\section{表格绘制}
表格也是浮动体,其排版可以参考图片的排版,引用的注意事项也类似。\par
\subsection{简单表格}
如果想绘制一个简单的表格,可以直接使用 \LaTeX 提供的表格语法,清单 \ref{list_绘制三行三列表格}所示是一个简单的三行三列表格的例子。

\begin{lstlisting} [language = tex,caption = 绘制三行三列表格,label = list_绘制三行三列表格]
\begin{table}[htbp] % 创建一个table,[htbp]表示浮动格式,即latex会自动找到最合适的位置放置该表格
  \centering
  \begin{tabular}{c c c} % tabular表示表格主体部分    
    \toprule[1pt] % 创建一条1pt的横线
    A & B & C \\  % 第一行,使用&分隔单元格,使用\\换行
    \hline        % 添加一条横线
    1 & 2 & 3 \\
    4 & 5 & 6 \\
    \toprule[1pt]
  \end{tabular}   % 结束tabular
\end{table}       % 结束table
\end{lstlisting}\par 
上述代码创建的表格如表 \ref{table_简单三线表格}所示。
\begin{table}[htbp] 
  \centering 
  \caption{三线表格}
  \begin{tabular}{c c c}
    \toprule[1pt]
    A & B & C \\  
    \hline       
    1 & 2 & 3 \\
    4 & 5 & 6 \\
    \toprule[1pt]
  \end{tabular}  
  \label{table_简单三线表格}
\end{table}

\subsection{excel2latex插件}
在表格过于复杂或者涉及数学计算时,可以使用excel编辑表格,再通过 \href{https://ctan.org/tex-archive/support/excel2latex}{\textcolor{blue}{excel2latex}} 插件转为\LaTeX 代码并将其复制到文档中。\par

\section{代码框}
Zafu的规范没有限制代码的表达方式,学科实例文件也并未提及(是觉得咱建筑学不需要用吧可能=. =)。本模板使用listings宏包实现代码框,样式 copy 自兰州大学兰朵儿提供的本科毕业论文 \LaTeX 模板。清单 \ref{list_代码}所示的代码实现代码框如 \ref{list_代码实现}所示。

\begin{lstlisting} [language = tex,caption = 代码,label = list_代码]  
  \begin{lstlisting} [language = tex,caption = 代码框,label = list_代码框]
    code..
    code...
  \textbackslash end{lstlisting}\par
  \end{lstlisting}\par 

\begin{lstlisting} [language = tex,caption = 代码实现,label = list_代码实现]
  code..
  code...
\end{lstlisting}\par


“[]”内的可选参数中,可以选择不同的编程语言(.cls文件内设置)。默认能选择tex、[Sharp]C、python、java、c++、matlab、XML。选择不同的语言会对对映的关键词进行识别,在代码框中显示不同的文本样式。注意:若使用 C\# 语言,可选参数应输入为“[language = {[Sharp]C}]”(建筑学相关能接触到最多的语言就是 python 和 C\# 咯)。


\section{数学公式}
作者写文章的时候没有用到数学公式,仅仅略懂,在此便不作详细介绍。\par
推荐参看教程:\href{https://www.bilibili.com/video/BV14g4y1q7pb?vd_source=0332e23098482db275098751af53ce78}{\textcolor{blue}{如何优雅的编辑数学公式?LaTeX公式入门}}\par
实现可视化数学公式输入的网页:\href{www.latexlive.com}{\textcolor{blue}{LaTeX公式编辑器}}

% 引用参考文献
\clearpage
\bibliographystyle{gbt7714-author-year}
\bibliography{ref}

\Thanks
{
  岁月匆匆,大学时光如白驹过隙,我即将踏入毕业设计的最后阶段,心潮澎湃,回首往昔,感慨万千。在此,我要向所有曾经帮助、陪伴过我的人们深深地致以诚挚的谢意。\par
  首先,我要感谢母校,是她为我搭建了知识的殿堂,为我提供了探索未来的舞台。五年时光,荟萃了她对我的呵护与培育,我将永怀感恩之心。\par
  感谢建筑学院的恩师们,你们的悉心教导和引领,让我在学术之路上不断探索,不断超越。你们是我学习的灯塔,照亮了前行的道路。
}



\end{document}

