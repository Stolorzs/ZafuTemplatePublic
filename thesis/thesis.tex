% \documentclass[a4paper]{article}
% \usepackage{ctex}
\documentclass[AutoFakeBold]{ZafuThesis}

\academy{风景园林与建筑学院}%学院
\session{2024}%第*届
\author{买一盒、松手}%作者姓名
\studentNumber{20190000000000000}%学号
\myClass{建筑学191}%班级
\teacher{建筑系全体教师}{教授/副教授/讲师}%指导老师
\date{\today}%日期,显示在封面与诚信承诺书页
% \title{{浙江农林大学本科生毕业设计模板 }}%标题单行
\title{{浙江农林大学本科生}{毕业设计说明书或论文模板}}%标题双行
\entitle{Architectural design of Heshang Police Station of Xiaoshan District `Public Security Bureau}%英文标题,显示在摘要页


\begin{document}
% %封面
\customCover

%自己的签名
\setsignature
  {
  \includegraphics[width=100pt]{figures/MaiYihe}
  \includegraphics[width=120pt]{figures/SongShou}
  }
% 承诺页
\makestatement


%目录
\customContent

\frontmatter
%中文摘要,关键词分号分隔
\ZhAbstract
{
  摘要内容主要介绍所研究的课题内容、提出主要结论及创新之处。摘要部分格式:黑体加粗五号段前空两个汉字字符,摘要内容楷体五号,不超过300个汉字。关键词部分格式:黑体加粗五号段前空两个汉字字符,关键词内容楷体五号,术语用分号隔开,数量一般为3-6个。Abstract:Times New Roman加粗五号段前空两个汉字字符;Abstract内容:Times New Roman五号。Key Words: Times New Roman加粗五号段前空两个汉字字符; Key Words内容:Times New Roman五号,术语用逗号隔开。
  
}
{关键词1;关键词2;关键词3;关键词4;关键词5}


%英文摘要,关键词用逗号分隔
\EnAbstract
{In this short article we will discuss about LATEX for your dissertation}
{keyword 1, keyword 2, keyword 3}


\mainmatter
\section{使用说明}
ZafuTemplate(浙江农林大学本科生毕业设计说明/论文模板与开题报告模板)是厌倦了低效的Word的某人完成毕业任务时心血来潮所作,现发布到网上供浙江农林大学本科毕业生免费使用,在使用前请仔细阅读下面的内容和注意事项。\par
学校和学院的要求存在一定的差异,据了解,甚至同一学院的不同专业对论文模板的要求也有所不同(Zafu什么时候能统一一下啊!)。总之,本模版尚存不足之处,欢迎反馈,更希望大家能帮忙一起完善。
\subsection{免责声明}
本模板格式参照2024届毕设或论文材料要求内建筑学专业的示例文件所作。'.cls'格式的文件内容由本人原创,部分代码借鉴学习自兰州大学兰朵儿所作的本科毕业论文模板\href{https://github.com/yuhldr/LZUThesis2020}{
  ( \textcolor{blue}{点击可访问其GitHub页面})}。\par
作者的初衷是减少后人花费在格式调整上的无用功,使学弟学妹们能够专注于文章内容的写作,从而提高本科生的毕业论文/设计水平。作者不保证本模板完全符合学科要求,因使用本模板产生的损失{\bfseries 由使用者自负},作者{\bfseries 不承担任何责任}!
\subsection{目录下的文件说明}
\begin{itemize}
  \item ".vscode"是VSCode的配置文件(若使用别的编辑器可以忽略)
  \item "2024届毕设或论文材料要求"文件夹内包含学校要求说明与建筑学专业的示例文档
  \item "thesis"内是毕业设计说明/论文的LaTex模板{\bfseries 主体文件}
  \item "ZafuResearchProposal"内是开题报告的LaTex模板{\bfseries 主体文件}
\end{itemize}
\par
主题文件中'.tex' 文件是 LaTeX 文档的源文件。它包含实际的文档内容和 LaTeX 命令,用于生成最终的 PDF 文件;'.cls'文件是 LaTeX 文档类文件,定义了文档的整体布局和样式(如有需要,请在其中自行修改格式)。它包含了一组命令和宏,用于规范文档的格式;'figures'文件夹下为要在文中展现的图片;'.bib' 文件是用于存储参考文献的数据库文件。
\subsection{安装和配置}
请自行根据操作系统对映选择安装TeX发行版,在此不过多赘述。推荐使用VSCode编辑器编译运行。\par
教程推荐:\href{https://www.bilibili.com/video/BV12m4y1D7PZ/?vd_source=dfa6f0143619fda15de458493344dd04}{\textcolor{blue}{LaTeX论文写作指南——以VSCode编辑器为例}}\par
本模板主页:\href{https://github.com/Stolorzs/ZafuTemplatePublic}{\textcolor{blue}{ZafuTemplatePublic}}

\subsection{LaTex生成的PDF转化DOCX}
许多导师不会使用PDF编辑器从而要求学生提交DOCX批阅,或者学院要求提交DOCX格式的文档,迫于上述现实因素,不得不研究将LaTeX导出的PDF转化为DOCX格式的方法。\par
\paragraph{转化方式}
使用 Adobe Acrobat DC 打开 LaTeX 生成的 PDF 文件,然后选择“另存为 DOCX”即可完成转换。作者测试了 Adobe Acrobat DC 2023 及以上版本,绝大多数的字体格式与图片排版都能在Word中对映上,效果较好,但未对带公式的转化(因为咱建筑学写论文很少用公式)进行测试。
\paragraph{注意事项}
在 MacOS 下编译生成的PDF不要导出到Windows操作系统下转化,在 Windows 下编译生成的 PDF 也不要导出到 MacOS 操作系统下转化,不然字体的格式会发生错误。若转化效果欠佳,可以考虑将 PDF 拆分成多组内容,分组进行转化再合并。如,将毕业论文拆分为封面、诚信承诺书、目录、摘要、正文主体几部分内容,分组转化为DOCX后在Word中进行合并。



\section{文字格式}
\subsection{参考规范}
请使用者自行参阅毕业当年的《浙江农林大学本科生毕业论文(设计)系列材料》与学科给出的示例文件。
\subsection{排版规范}
\LaTeX 是一种按照层级排列的标记语言,使用不同的符号来区分不同的内容,以正文部分为例,一个具有三层标题的正文内容片段主要由以下代码组成:
\begin{lstlisting}[]
\section{一级标题}  //花括号"{}"内撰写标题的内容
  正文第一段\par  //使用\par开启一个新的段落
  正文第二段
  \subsection{二级标题}
    正文第一段\par
    正文第二段
    \subsubsection{三级标题}
      正文第一段\par
      正文第二段
//注意:1)标题的序号会自动生成,无需在标题内填写序号。
//      2)正文默认首行缩进两字符,如果想要某一段不具有首行缩进,可以在段首加上\noindent
\end{lstlisting}

\par
使用 \LaTeX 撰写论文的初衷是抛弃繁琐的word格式调整工作,把更多的心思放在文章内容本身,使用者只需在thesis.tex中写下文本,thesis.pdf就会按照设定好的格式生成。然而,由于学校各学院格式不统一,使用者还是需要对设定格式有一定的了解,在遇到格式不统一时,适当对本模板的格式进行调整。\par

本模板参照《浙江农林大学本科生毕业论文(设计)系列材料(2024版)》格式要求,对论文的封面、诚信承诺书、目录、中英文摘要、正文、参考文献以及致谢等部分的格式进行了设置。

\subsubsection{章节标题和序号}

章节标题和序号参考学校对于\textbf{理工科类}论文的格式要求,以阿拉伯数字为序号,采用数字和"."的组合。如需使用\textbf{文科格式},请自行修改。
\paragraph{一级标题}采用楷体加粗四号居中,序号与文字间空1个汉字字符,段前、段后6磅(或1行);
\paragraph{二级标题}采用黑体加粗小四,段前空两个汉字字符,序号与文字间空1个汉字字符,段前、段后3磅(或0.5行);
\paragraph{三级标题}采用黑体五号,段前空两个汉字字符,序号与文字间空1个汉字字符,段前、段后3磅(或0.5行);
\paragraph{三级以下标题}本模板暂未实现。

\par 本模板对于标题格式的设置使用了titlesec宏包,以一级标题为例,具体设置如下:
\begin{lstlisting}
  \titleformat{\section}{\centering\zihao{4}\bfseries\CJKfamily{zhkai}}{\thesection}{0.5em}{}
  \titlespacing*{\section}{0pt}{9.75pt}{9.75pt}
  // \titleformat用于设置标题字体及段落格式 
  // \titleformat{command}[shape]{format}{label}{sep}{before}[after]
  // \titlespacin*用于设置标题的与其他部分相隔的距离
  // \titlespacing*{command}{left}{beforesep}{aftersep}[right]
\end{lstlisting}
\paragraph{注意} 关于\textbackslash titleformat和\textbackslash titlespacing*的具体用法请参阅:\href{https://static.latexstudio.net/wp-content/uploads/2016/12/titlesec_c.pdf}{\textcolor{blue}{titlesec文档(第5,6页)}}。另外,word与 \LaTeX 的度量单位并不完全一致,如果对于排版的精确度要求很高,请进行合理的转换。
\par 本模板还提供了带有标题的段落格式,具体用法为“\textbackslash paragraph\{段落标题\}段落内容”。

\subsubsection{ctexart宏包}
本模板使用了中文汉字支持宏包ctexart,模板预设中文字体有四种:宋体、黑体、仿宋、楷书,具体命令如下:
\begin{table}[htbp]
  \centering
  \begin{tabular}{c c c c}
    \toprule[1pt]
    宋体 & 黑体 & 仿宋 & 楷书 \\
    \hline
    \textbackslash songti & \textbackslash heiti & \textbackslash fangsong & \textbackslash kaishu \\
    \toprule[1pt]
  \end{tabular}
\end{table}
\par ctexart宏包还提供了控制字号的命令,\textbackslash zihao\{5\}为五号,\textbackslash zihao\{-4\}为小四,但请注意:仅当标准字体命令为\textbackslash normalsize 时有这样的对应结果,具体对应关系请参阅:\href{https://mirrors.ibiblio.org/CTAN/language/chinese/ctex/ctex.pdf}{\textcolor{blue}{CTEX 宏集手册-5.1(p8)}}。


\begin{table}[htbp]
  \centering
  \begin{tabular}{c c c c c}
    \toprule[1pt]
    & \multicolumn{2}{c|}{\textbackslash zihao\{5\}} & \multicolumn{2}{c}{\textbackslash zihao\{-4\}}  \\

   字体命令 & 字号 &\multicolumn{1}{c|}{bp}  & 字号 & bp \\
    \hline
    \textbackslash normalsize & 五号 & 10.5  &  小四 & 12 \\
    \toprule[1pt]
  \end{tabular}
\end{table}

\par
想要了解使用ctexart宏包设置字体格式的具体方法请参阅:\href{https://mirrors.ibiblio.org/CTAN/language/chinese/ctex/ctex.pdf}{\textcolor{blue}{CTEX 宏集手册-第7节(p15)}}。

\subsubsection{字体格式}
下面介绍了一些 \LaTeX  原生支持的字体格式设置方法,如字体颜色、文字加粗、斜体等。


\paragraph{字体颜色} 使用“\textbackslash textcolor\{颜色\}\{文字\}”设置“文字”的颜色,其中“颜色”部分可以使用预设,也可以通过RGB色值设置自定义颜色: “\textbackslash textcolor[rgb]\{0.25, 0.5, 0.75\}\{文字\}”。下面是一些常用的颜色预设:
\begin{table}[htbp]
  \centering
  \begin{tabular}{c c c}
    \toprule[1pt]
    关键字 & 对应颜色\\
    \hline
    red &  \textcolor{red}{红色} \\
    blue &  \textcolor{blue}{蓝色} \\
    green &  \textcolor{green}{绿色} \\
    purple &  \textcolor{purple}{紫色} \\
    cyan &  \textcolor{cyan}{青色} \\


    \toprule[1pt]
  \end{tabular}
\end{table}
\paragraph{文字样式} \LaTeX 提供了设置字体样式的命令,每种样式对应两组命令。其中如“\textbackslash bfseries”,为全局命令,通常用于在.cls中定义样式,或者写成“\{ \textbackslash bfseries 内容 \}”这样的形式;而与其对应的“\textbackslash textbf”命令则用于局部修改样式,写成“\textbackslash textbf\{内容\}”这样的形式。下面列举了一些常用的字体样式命令:
\begin{table}[htbp]
  \centering
  \begin{tabular}{c c c c}
    \toprule[1pt]
    全局命令 & 局部命令 & 英文 & 中文 \\
    \hline
    \textbackslash mdseries  & \textbackslash textmd\{...\} & \textmd{medium}& \textmd{正常粗细} \\
    \textbackslash bfseries  & \textbackslash textbf\{...\} & \textbf{Bold}& \textbf{粗体} \\
    \textbackslash itshape  & \textbackslash textit\{...\} & \textit{italic}& \textit{意大利斜体} \\
    \textbackslash slshape  & \textbackslash textsl\{...\} & \textsl{slanted}& \textsl{倾斜体} \\
    \toprule[1pt]
  \end{tabular}
\end{table}
\subsubsection{文献引用}
学校与学院对参考文献的引用格式不统一,本模板按照学院对文内引用标注的要求“(作者,年份)”,介绍两种常用方法:
\paragraph{第一种是使用 \textbackslash bibitem 手动填写}
\begin{lstlisting} 
\begin{thebibliography}{99}
  \bibitem{ref1} reference1 
  \bibitem{ref2} reference2
  \bibitem{ref3} reference3
\end{thebibliography}
//注意:1)需要在.tex文件中添加\bibliographystyle{gbt7714-numerical}
//      2)reference1处自行按照标准填写参考文献引用格式
//      3)使用\cite{ref1}增加文内引用标注,但其格式只能是[1]形式,其
//         他格式只能手动输入 如(作者, 年份)
\end{lstlisting}

\paragraph{第二种是使用.bib文件自动生成}即在.bib文件中加入文献的引用,然后在tex中自动生成,具体内容在可以在各个文献的官网上导出,格式大概如下:
\begin{lstlisting} 
@book{ref1,
author = {author}, 
title = {title},
publisher = {},
year = 2018,
}
//注意:1)需要在.tex文件中添加\bibliographystyle{gbt7714-author-year}
//         且需注释掉:\bibliographystyle{gbt7714-numerical}
//      2)gbt7714-author-year表示使用(作者,年份)的文内引用标注格式
//      3)每一个参考文献都需在文内使用\citep{}进行引注,否则无法生成
//      4)在.tex文件的合适位置,添加\bibliography{ref},即可自动生成
\end{lstlisting}



\section{图片插入}
首先需要将图片放在figures目录下,然后使用代码生成。
\begin{lstlisting}
  \begin{figure}[h] // begin{figure}开始一个图表 [h]表示放在当前位置
    \centering      // \centering表示图表整体居中
    \includegraphics[width=100pt]{figures/MaiYihe} 
                    // []内写图片宽度,{}内写图片的相对路径或绝对路径
    \caption{这是一个居中的图片} // 图片的脚注
  \end{figure}
\end{lstlisting}
\par
\begin{figure}[h]
  \centering\includegraphics[width=100pt]{figures/MaiYihe}
  \caption{这是一个居中的图片}
\end{figure}
\section{表格绘制}
\subsection{简单的表格}
如果想绘制一个简单的表格,可以直接使用 \LaTeX 提供的表格语法,下面举一个简单的三行三列表格的例子:
\begin{lstlisting}
\begin{table}[htbp] //创建一个table,[htbp]表示浮动格式,即latex会自动找到最合适的位置放置该表格
  \centering        //表格整体居中
  \begin{tabular}{c c c} //tabular表示表格主体部分
    //1){c c c}表示表格有三列,均为居中(r代表右对齐,l代表左对齐)
    //2){c|c|c}表示为具体的列之间加上格线
    \toprule[1pt] //创建一条1pt的横线
    A & B & C \\  //第一行,使用&分隔单元格,使用\\换行
    \hline        //添加一条横线
    1 & 2 & 3 \\
    4 & 5 & 6 \\
    \toprule[1pt]
  \end{tabular}   //结束tabular
\end{table}       //结束table
\end{lstlisting}
\par 以上代码创建的表格如下:
\begin{table}[htbp] 
  \centering 
  \begin{tabular}{c c c}
    \toprule[1pt] 
    A & B & C \\  
    \hline       
    1 & 2 & 3 \\
    4 & 5 & 6 \\
    \toprule[1pt]
  \end{tabular} 
\end{table} 
\subsection{excel2latex插件}
在表格过于复杂或者涉及数学计算时,建议使用excel编辑表格,再通过excel2latex插件转为\LaTeX 代码并将其复制到文档中。
\par
下载地址:https://ctan.org/tex-archive/support/excel2latex/

\section{代码框}

\section{数学公式}




% % 引用参考文献
\clearpage
% \bibliographystyle{gbt7714-numerical}
% \bibliographystyle{gbt7714-author-year}

% \bibliography{ref}

\Thanks
{
  岁月匆匆,大学时光如白驹过隙,我即将踏入毕业设计的最后阶段,心潮澎湃,回首往昔,感慨万千。在此,我要向所有曾经帮助、陪伴过我的人们深深地致以诚挚的谢意。\par
  首先,我要感谢母校,是她为我搭建了知识的殿堂,为我提供了探索未来的舞台。五年时光,荟萃了她对我的呵护与培育,我将永怀感恩之心。\par
  感谢建筑学院的恩师们,你们的悉心教导和引领,让我在学术之路上不断探索,不断超越。你们是我学习的灯塔,照亮了前行的道路。
}



\end{document}

