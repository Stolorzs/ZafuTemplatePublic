% \documentclass[a4paper]{article}
% \usepackage{ctex}
\documentclass[AutoFakeBold]{ZafuThesis}

\setmainfont{Times New Roman}
\setCJKmainfont{宋体}

\academy{风景园林与建筑学院}%学院
\session{2024}%第*届
\author{买一盒、路完成}%作者姓名
\studentNumber{20190000000000000}%学号
\myClass{建筑学191}%班级
\teacher{建筑系全体教师}{教授/副教授/讲师}%指导老师
\date{\today}%日期,显示在封面与诚信承诺书页
% \title{{浙江农林大学本科生毕业设计 }}%标题单行
\title{{浙江农林大学本科生毕业设计 }{说明书(论文)模板}}%标题双行
\entitle{Architectural design of Heshang Police Station of Xiaoshan District Public Security Bureau}%英文标题,显示在摘要页


\begin{document}

% %封面
\customCover

% 承诺页
\makestatement
%自己的签名
\mysignature{\includegraphics[width=100pt]{zafu}}

%目录
\customContent

\frontmatter
%中文摘要,关键词分号分隔
\ZhAbstract
{
  摘要内容主要介绍所研究的课题内容、提出主要结论及创新之处。摘要部分格式:黑体加粗五号段前空两个汉字字符,摘要内容楷体五号,不超过300个汉字。关键词部分格式:黑体加粗五号段前空两个汉字字符,关键词内容楷体五号,术语用分号隔开,数量一般为3-6个。Abstract:Times New Roman加粗五号段前空两个汉字字符;Abstract内容:Times New Roman五号。Key Words: Times New Roman加粗五号段前空两个汉字字符; Key Words内容:Times New Roman五号,术语用逗号隔开。
  
}
{关键词1;关键词2;关键词3;关键词4;关键词5}


%英文摘要,关键词用逗号分隔
\EnAbstract
{In this short article we will discuss about LATEX for your dissertation}
{keyword 1, keyword 2, keyword 3}



\mainmatter
\section{模版介绍与注意事项}
\subsection{模版说明}
ZafuTemplate是作者本人自己写毕业论文时心血来潮所作,现发布到网上供浙江农林大学本科毕业生免费使用,在使用前请仔细阅读下面的内容和注意事项。\par
注意:本模版尚存许多不足之处,欢迎大家反馈,更希望大家能帮忙一起完善这个模版。
\subsection{下载安装}
ZafuTemplate 主页:...。
\subsection{文件说明}
在/ZafuTemplate目录中含有以下几个主要的文件,使用者可以根据自己的需要进行修改,从而完成自己的毕业论文:
\begin{itemize}
  \item these.tex:论文的内容,撰写论文请在该文件下。
  \item ZafuThesis.cls:用来控制论文格式的文件,如果有需要,可以自行修改格式。
\end{itemize}

\subsection{参考文献}
\subsection{编译注意事项}
\subsection{免责声明}
本模板依据《浙江农林大学本科生毕业论文(设计)系列材料(2022版)》编写,作者希 望能给使用者写作论文带来方便。然而,作者不保证本模板完全符合学校要求,也 不对由此带来的风险和损失承担任何责任。
\section{格式要求}
\subsection{论文格式参考}
本部分参阅:《浙江农林大学本科生毕业论文(设计)系列材料(2022版)》,以下将罗列出学校论文具体格式要求,供模版使用者进行参阅对比\par
\subsection{封面}
参照学校提供的统一封面,内容包括:学院名称,题目,学生姓名、学号,专业班级,指导教师姓名、职称。所有内容要居中。本模版封面按照学校提供的封面所制,排版上相同,但在打印上略有差别,使用封面时请自行斟酌。
\subsection{目录}
目录独立成页,包括论文中全部章、节的标题及页码。设计图纸要有标号。
\subsubsection{理工科类}

采用的三级标题序次结构有以下一种:\par
序次:1、1.1、1.1.1……\par
目录内容采用宋体五号。\par
注意:本模版目录格式按照为理工科要求所设,若需使用文科目录格式请自行修改。
\subsubsection{文科类}
目前采用的三级标题序次结构有以下两种:\par
第一种序次:第一章、第一节、一……\par
第二种序次:一、(一)、1……\par
目录采用宋体加粗四号居中,目录内容采用宋体五号,数字、字母采用Time New Roman五号。
\subsubsection{题目}
黑体三号居中,不宜超过20个汉字。
\subsubsection{摘要、关键词}
见模版摘要部分要求
\subsubsection{论文文体}
文本主体一般包括引言(或称前言、序言等)、正文与结论三部分。引言宣示课题的“来龙”,应说明本课题的意义、目的、主要研究内容、范围及应解决的问题。也可以不用引言,直接从第一章写起。正文是毕业论文(设计)的核心。结论部分可以用“结束语”、“结语”等标题来表示,也可以不用标题表示。\par



\section{图片插入}
\section{表格绘制}
\subsection{excel2latex插件}
https://ctan.org/tex-archive/support/excel2latex/
\subsection{格式调整}


\section{代码插入 }




% % 引用参考文献
% \clearpage
% \bibliographystyle{gbt7714-author-year}
% \bibliography{ref}
% \Reference{
% }
\Thanks
{
  岁月匆匆,大学时光如白驹过隙,我即将踏入毕业设计的最后阶段,心潮澎湃,回首往昔,感慨万千。在此,我要向所有曾经帮助、陪伴过我的人们深深地致以诚挚的谢意。\par
  首先,我要感谢母校,是她为我搭建了知识的殿堂,为我提供了探索未来的舞台。五年时光,荟萃了她对我的呵护与培育,我将永怀感恩之心。\par
  感谢建筑学院的恩师们,你们的悉心教导和引领,让我在学术之路上不断探索,不断超越。你们是我学习的灯塔,照亮了前行的道路。
}



\end{document}

